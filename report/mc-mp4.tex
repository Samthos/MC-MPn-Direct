\documentclass{revtex4}
\usepackage{amsmath}

\begin{document}
Direct sampling of a distribution is desirable as the sampling is free from correlation and due to the assurance of sampling all space appropriately.

\section{Direct Sampling}
  In previous work, we have proposed an importance function suitable for the stochastic integration of perturbation corrections to the energy.
  This importance function has the form
  \begin{align}
    w(\mathbf{r}_1, \mathbf{r}_2)
    &= \frac{1}{N_g}\sum_{AB}\sum_{i}^{N_A}\sum_j^{N_B}\frac{g_{Ai}(\mathbf{r}_1)g_{Bj}(\mathbf{r}_2)}{r_{12}},
    \label{eq:imp}
  \end{align}
  where $A, B$ label atoms and $N_A$ count the number of basis functions on atoms $A$.
  \begin{equation}
    g_{Ai}(\mathbf{r}) = c_{Ai} e^{-\alpha_{Ai}|\mathbf{r}-\mathbf{R}_A|^2},
  \end{equation}
  where $c_{Ai}$ and $\alpha_{Ai}$ are coefficient and exponent of the $i$th basis function on atom $A$.
  The normalization constant for the importance function is
  \begin{equation}
    N_g = \sum_{AB}\sum_{i}^{N_A}\sum_j^{N_B} N_{Ai,Bj},
  \end{equation}
  where 
  \begin{equation}
     N_{Ai,Bj} = \int \cdots \int \mathrm{d}\mathbf{r}_1 \mathrm{d}\mathbf{r}_2 \frac{g_{Ai}(\mathbf{r}_1)g_{Bj}(\mathbf{r}_2)}{r_{12}}.
    \label{eq:impTerm}
  \end{equation}

  To sample Eq.\ (\ref{eq:imp}) directly, first a single summand with index $Ai, Bj$ in Eq.\ (\ref{eq:imp}) is chosen with probability $N_{Ai,Bj}/N_g$.
  Then, the coordinates for an electron-pair are generated according to the probability distribution function
  \begin{align}
     w_{Ai,Bj}(\mathbf{r}_1, \mathbf{r_2}) = \frac{1}{N_{Ai, Bj}} \frac{g_{Ai}(\mathbf{r}_1)g_{Bj}(\mathbf{r}_2)}{r_{12}}.
     \label{eq:sgl_imp}
  \end{align}

  To generate a sample according to $w_{Ai,Bj}(\mathbf{r}_1, \mathbf{r}_2)$, the six coordinates of the electron pair are decoupled into 4 independent coordinates and one pair of coupled
  coordinates. The decoupling is performed by two coordinate transformations.
  As the cumulative probability distribution functions for SEVERAL of the decouple coordinates need to be calculated, the Jacobian of each coordinate transformation is noted.
  
  We first shift the Gaussian functions such that their centers are symmetric about the origin and aligned along the $z$-axis.
  This rotation chosen to specifically shift the Gaussian with center $\mathbf{R}_A$ to be along the positive $z$-axis.
  The transformation is coordinates is given by
  \begin{align}
    x_1 &= \bar{x}_1 + (X_A + X_B)/2 \\
    y_1 &= \bar{y}_1 + (Y_A + Y_B)/2 \\
    z_1 &= \bar{z}_1 + (Z_A + Z_B)/2 \\
    x_2 &= \bar{x}_2 + (X_A + X_B)/2 \\
    y_2 &= \bar{y}_2 + (Y_A + Y_B)/2 \\
    z_2 &= \bar{z}_2 + (Z_A + Z_B)/2,
  \end{align}
  where $x_1...$ are the ordinal coordinates, $\bar{x}_1...$ are the shifted coordinates, $x_a...$ are the centers of the Gaussian functions in the original coordinate system.
  The centers of the Gaussian in the transformed coordinate system are described by
  \begin{align}
    (\bar{X}_t, \bar{Y}_t, \bar{Z}_t) &= (0, 0, |\mathbf{R}_B - \mathbf{R}_A| / 2).
  \end{align}
  As this is transformation is a translation and a rotation, its Jacobian is trivially unity.
  After the transformation Eq.\ (\ref{eq:sgl_imp}) becomes
  \begin{align}
    w_{Ai,Bj}(\bar{\mathbf{r}}_1, \bar{\mathbf{r}}_2) = 
    c_{\alpha} c_{\beta}
    \frac{
      \exp(-\alpha(\bar{x}_1^2 + \bar{y}_1^2 + (\bar{z}_1 - Z_t)^2))
      \exp(-\beta(\bar{x}_2^2 + \bar{y}_2^2 + (\bar{z}_2 + Z_t)^2))
    }
    {
      \sqrt{(\bar{x}_1 - \bar{x}_2)^2 + (\bar{y}_1 - \bar{y}_2)^2 + (\bar{z}_1 - \bar{z}_2)^2},
    }
  \end{align}
  where for compactness we have set $(c_\alpha, c_\beta, \alpha, \beta) = (c_{Ai}, c_{Bj}, \alpha_{Ai}, \alpha_{Bj})$.

%  \begin{align}
%    \bar{\mathbf{r}}_1 &= \mathbf{R}(\phi, y)\mathbf{R}(\theta, z)\tilde{\mathbf{r}}_1 \\
%    \bar{\mathbf{r}}_2 &= \mathbf{R}(\phi, y)\mathbf{R}(\theta, z)\tilde{\mathbf{r}}_2 \\
%  \end{align}
%  where $\mathbf{R}(t, p)$ is the rotation matrix corresponding to a rotation of $t$ radians about axis $p$.

  The second transformation introduces a center of mass like coordinates 
  \begin{align}
    \bar{x}_1 &= \tilde{x} - \tilde{r} \cos(\tilde{\theta}) \sin(\tilde{\phi}) / \alpha \\
    \bar{y}_1 &= \tilde{y} - \tilde{r} \sin(\tilde{\theta}) \sin(\tilde{\phi}) / \alpha \\
    \bar{z}_1 &= \tilde{z} - \tilde{r} \cos(\tilde{\phi}) / \alpha \\
    \bar{x}_2 &= \tilde{x} + \tilde{r} \cos(\tilde{\theta}) \sin(\tilde{\phi}) / \beta \\
    \bar{y}_2 &= \tilde{y} + \tilde{r} \sin(\tilde{\theta}) \sin(\tilde{\phi}) / \beta \\
    \bar{z}_2 &= \tilde{z} + \tilde{r} \cos(\tilde{\phi}) / \beta,
  \end{align}
  where $(\tilde{x}, \tilde{y}, \tilde{z})$ is the center of the coordinate pair, $\tilde{r}$ is half the inter-electronic distance, and $(\tilde{\theta}, \tilde{\phi})$ describe the orientation of
  the electrons.
  The Jacobian of this transformation is
  \begin{align}
    \frac{r^2 (\alpha +\beta )^3 \sin (\phi )}{8 \alpha ^3 \beta ^3}.
  \end{align}

  \begin{align}
  w_{Ai,Bj}(\bar{\mathbf{r}}_1, \bar{\mathbf{r}}_2) =& 
  \mathrm{N}\left(\tilde{x}; 0, \frac{1}{\sqrt{2(\alpha + \beta)}}\right)
  \mathrm{N}\left(\tilde{y}; 0, \frac{1}{\sqrt{2(\alpha + \beta)}}\right)
  \mathrm{N}\left(\tilde{z}; \frac{\bar{z}_t (\alpha -\beta )}{2 (\alpha +\beta )}, \frac{1}{\sqrt{2(\alpha + \beta)}}\right) \nonumber \\
  & \times \mathrm{U}\left(\tilde{\theta}; 0, 2 \pi\right)
    \mathrm{W}\left(\tilde{r}, \tilde{\phi}; \bar{z}_t, \alpha, \beta\right)
  \end{align}

  \begin{align}
    \mathrm{N}\left(x; \mu, \sigma\right) = \frac{1}{\sqrt{2\pi \sigma^2}} e^{\frac{(x-\mu)^2}{2 \sigma^2}}
  \end{align}

  \begin{align}
    \mathrm{U}\left(\theta; a, b\right) = \frac{1}{b - a} \hspace{5mm} b>a
  \end{align}

  If $\bar{z}_t$ is not equal zero, then
  \begin{align}
    \mathrm{W}\left(\tilde{r}, \tilde{\phi}; \bar{z}_t, \alpha, \beta\right) = 
  \end{align}

  \begin{align}
    \mathrm{W}_{r}\left(\tilde{r}; \bar{z}_t, \alpha, \beta\right) = 
  \end{align}

  Run Halley's method with starting point $E[r]$

  Generate phi according to 
  \begin{align}
    \mathrm{W}\left(\tilde{\phi} | \tilde{r}; \bar{z}_t, \alpha, \beta\right) = 
  \end{align}
  With Halley's method

  If $\bar{z}_t$ is equal to zero, i.e. $A$ and $B$ are the same atom then
  \begin{align}
    \mathrm{W}\left(r, \phi; 0, \alpha, \beta\right) = W_{r} W_{\phi}
  \end{align}

  \begin{align}
    \mathrm{W}_{r}\left(r; \alpha, \beta\right)
  \end{align}

  \begin{align}
    \mathrm{W}_{\phi}\left(\phi\right) = \frac{\mathrm{Sin}(\phi)}{2}
  \end{align}

\section{Control Variates}

\section{Proof of Two Step Sampling}
  \begin{align}
    \mathrm{P}(X=x) = \frac{1}{F}\sum_{i=1}f_i(x)
  \end{align}

  \begin{align}
    F = \sum_{i=1} F_i
  \end{align}

  \begin{align}
    F_i = \int f_i(x) \mathrm{d}x
  \end{align}

  \begin{align}
    \omega_i = \frac{F_i}{F}
  \end{align}

  \begin{align}
    U_i = \sum_{k=1}^{i} \omega_i
  \end{align}

  \begin{align}
    L_i = U_i - \omega_i 
  \end{align}

  \begin{align}
    \mathrm{P}(X=x, Y=y) = \sum_{i=1} \frac{f_i(x)}{F_i} H(U_i - y) H(y - L_i)
  \end{align}

  \begin{align}
    H(y) = 
    \left\{ \begin{array}{cc} 
    0 & \hspace{5mm} y<0 \\
    1 & \hspace{5mm} y>=1 \\
    \end{array} \right.
  \end{align}

  \begin{align}
    \mathrm{P}(X=x) &= \int_0^1 \mathrm{P}(X=x, Y=y) \mathrm{d}y \\
    &= \int_0^1 \sum_{i=1} \frac{f_i(x)}{F_i} H(U_i - y) H(y - L_i) \mathrm{dy} \\
    &= \sum_{i=1} \frac{f_i(x)}{F_i} \int_0^1 H(U_i - y) H(y - L_i) \mathrm{dy} \\
    &= \sum_{i=1} \frac{f_i(x)}{F_i} (U_i - L_i) \\
    &= \sum_{i=1} \frac{f_i(x)}{F_i} (\omega_i) \\
    &= \sum_{i=1} \frac{f_i(x)}{F_i} \frac{F_i}{F} \\
    &= \sum_{i=1} \frac{f_i(x)}{F}
  \end{align}

  \begin{align}
    \mathrm{P}(X=x|Y=y) = f_i \hspace{5mm} L_i < y < U_i
  \end{align}
\end{document}
